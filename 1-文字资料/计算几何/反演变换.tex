\textbf{反演变换 (Inversion)}

\textbf{定义}

给定平面上一点 \(O\)(反演中心)和常数 \(R > 0\)(反演半径)。对于平面上任意异于 \(O\) 的点 \(P\),其反演点 \(P'\) 满足:
\begin{enumerate}
    \item \(P'\) 在射线 \(OP\) 上;
    \item \(|OP| \cdot |OP'| = R^2\)。
\end{enumerate}
点 \(O\) 称为反演中心,圆 \(C(O, R)\) 称为反演圆。

\textbf{性质}

\begin{itemize}
    \item \textbf{对合性}:\((P')' = P\)。
    \item \textbf{反演圆上的点}:反演圆上的点反演后位置不变。
    \item \textbf{直线与圆的变换}:
    \begin{itemize}
        \item \textbf{过反演中心的直线} \(\to\) 自身(直线)。
        \item \textbf{不过反演中心的直线} \(\to\) 过反演中心的圆。
        \item \textbf{过反演中心的圆} \(\to\) 不过反演中心的直线。
        \item \textbf{不过反演中心的圆} \(\to\) 不过反演中心的圆。
    \end{itemize}
    \item \textbf{保角性}:反演变换保持曲线间的夹角大小和方向不变(反形保角)。
    \item \textbf{距离关系}:
    \[ |A'B'| = \frac{R^2}{|OA| \cdot |OB|} |AB| \]
\end{itemize}

\textbf{应用}

\begin{itemize}
    \item 将“圆与圆相切”的问题转化为“圆与直线相切”或“直线与直线平行”的问题。
    \item 处理多个圆共点的问题,选取交点为反演中心,可将这些圆转化为直线。
    \item 托勒密定理的证明。
\end{itemize}
