\textbf{曼哈顿距离与切比雪夫距离}

\textbf{定义}

对于平面上两点 \(A(x_1, y_1)\) 和 \(B(x_2, y_2)\):

\begin{itemize}
    \item \textbf{曼哈顿距离}:\(d_{Manhattan}(A, B) = |x_1 - x_2| + |y_1 - y_2|\)
    \item \textbf{切比雪夫距离}:\(d_{Chebyshev}(A, B) = \max(|x_1 - x_2|, |y_1 - y_2|)\)
\end{itemize}

\textbf{坐标变换}

\begin{itemize}
    \item \textbf{曼哈顿 \(\to\) 切比雪夫}:
    
    将原坐标 \((x, y)\) 变换为 \((x', y') = (x+y, x-y)\)。
    
    原坐标系下两点的曼哈顿距离等于新坐标系下两点的切比雪夫距离:
    \[ |x_1 - x_2| + |y_1 - y_2| = \max(|x'_1 - x'_2|, |y'_1 - y'_2|) \]

    \item \textbf{切比雪夫 \(\to\) 曼哈顿}:
    
    将原坐标 \((x, y)\) 变换为 \((x', y') = (\frac{x+y}{2}, \frac{x-y}{2})\)。
    
    原坐标系下两点的切比雪夫距离等于新坐标系下两点的曼哈顿距离。
\end{itemize}

\textbf{应用}

曼哈顿距离下的问题往往难以直接处理绝对值求和,通过变换为切比雪夫距离,可以将 \(x\) 轴和 \(y\) 轴的限制解耦(独立计算),从而利用数据结构或排序分别维护。
