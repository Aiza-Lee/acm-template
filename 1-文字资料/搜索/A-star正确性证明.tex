\textbf{A* 算法正确性证明}

\textbf{定义}

\begin{itemize}
    \item $g(n)$: 从起点到节点 $n$ 的实际代价。
    \item $h(n)$: 从节点 $n$ 到目标节点的估计代价(启发式函数)。
    \item $f(n) = g(n) + h(n)$: 经过节点 $n$ 到达目标的估计总代价。
    \item $C^*$: 从起点到目标的最优路径代价。
\end{itemize}

\textbf{可采纳性 (Admissibility)}

若对于所有节点 $n$,都有 $0 \le h(n) \le h^*(n)$,其中 $h^*(n)$ 为从 $n$ 到目标的实际最优代价,则称 $h(n)$ 是可采纳的。

\textbf{定理}

如果 $h(n)$ 是可采纳的,那么 A* 树搜索算法(Tree Search)保证能找到最优解。

\textbf{证明(反证法)}

假设 A* 算法终止于一个非最优的目标节点 $G_2$,即 $f(G_2) = g(G_2) > C^*$。

设 $G^*$ 为最优目标节点。在 A* 终止前,最优路径上必然存在一个未被扩展的节点 $n$ 在 OPEN 表中。

对于该节点 $n$:
\[ f(n) = g(n) + h(n) \]

由于 $h$ 是可采纳的,即 $h(n) \le h^*(n)$,且 $n$ 在最优路径上,故:
\[ f(n) \le g(n) + h^*(n) = C^* \]

因为 A* 选择了扩展 $G_2$ 而不是 $n$,根据 A* 的选择策略(选择 $f$ 值最小的节点),必有:
\[ f(G_2) \le f(n) \]

结合上述不等式:
\[ f(G_2) \le f(n) \le C^* \]

这与假设 $f(G_2) > C^*$ 矛盾。因此假设不成立,A* 必定找到最优解。

\textbf{一致性 (Consistency) / 单调性}

对于图搜索(Graph Search,即维护 CLOSED 表),仅有可采纳性是不够的。需要满足一致性条件:
\[ h(n) \le c(n, n') + h(n') \]
其中 $n'$ 是 $n$ 的后继节点,$c(n, n')$ 是边权。

若 $h$ 满足一致性,则 $f(n)$ 沿任意路径单调递增,A* 第一次扩展到某节点时即为该节点的最优路径。
