\textbf{竞赛图 (Tournament Graph)}

\textbf{定义}
$n$ 个顶点的有向图,对于任意两个不同的顶点 $u, v$,恰有一条边连接 $u \to v$ 或 $v \to u$。竞赛图可视作完全图给每一条边定向后的结果。

\textbf{兰道定理 (Landau's Theorem)}
一个非负整数序列 $s_1 \le s_2 \le \dots \le s_n$ 是某个 $n$ 阶竞赛图的出度序列(Score Sequence),当且仅当满足以下条件:
\[
\sum_{i=1}^k s_i \ge \binom{k}{2}, \quad \forall 1 \le k < n, \quad \text{且} \quad \sum_{i=1}^n s_i = \binom{n}{2}
\]
\textbf{推论(SCC 分布)}:
将出度序列升序排列后,若在下标 $k_1, k_2, \dots, k_m = n$ 处满足等号 $\sum_{i=1}^{k_j} s_i = \binom{k_j}{2}$,则这些位置是强连通分量(SCC)的分割点。
具体来说,缩点后的 DAG 是一条链 $C_1 \to C_2 \to \dots \to C_m$,其中 $C_1$ 包含 $s_1 \dots s_{k_1}$ 对应的点,$C_2$ 包含 $s_{k_1+1} \dots s_{k_2}$ 对应的点,以此类推。
这意味着:\textbf{出度序列唯一确定了竞赛图的强连通分量大小及拓扑结构}。

\textbf{核心性质}
\begin{enumerate}
    \item \textbf{哈密顿路 (Redei's Theorem)}:任意竞赛图都存在一条哈密顿路径。
    \item \textbf{强连通性 (Camion's Theorem)}:竞赛图强连通 $\iff$ 存在哈密顿回路。
    \item \textbf{泛圈性 (Moon's Theorem)}:若 $n$ 阶竞赛图强连通,则对于任意顶点 $v$ 和任意 $k \in [3, n]$,都存在包含 $v$ 的长度为 $k$ 的回路。
    \begin{itemize}
        \item \textbf{推论}:强连通竞赛图包含 $3, 4, \dots, n$ 长度的环。
    \end{itemize} 
    \item \textbf{传递性}:竞赛图不含环 $\iff$ 它是传递的($u \to v, v \to w \implies u \to w$) $\iff$ 出度序列为 $0, 1, \dots, n-1$(即 DAG,具有唯一的拓扑序)。
    \item \textbf{SCC缩点形态}:竞赛图缩点后的 DAG 呈\textbf{线性链状}($C_1 \to C_2 \to \dots \to C_m$),即任意 $u \in C_i, v \in C_j (i < j)$,都有 $u \to v$。
    \begin{itemize}
        \item 这种结构导致竞赛图的拓扑排序(Topological Sort)是唯一的(如果不考虑同一SCC内部顺序)。
    \end{itemize}
\end{enumerate}

\textbf{计数相关}
\begin{enumerate}
    \item \textbf{三元环计数}:
    $n$ 阶竞赛图中三元环的数量等于总的三元组数量 $\binom{n}{3}$ 减去非循环三元组的数量。非循环三元组必有一个点出度为 2(在该三元组诱导子图中)。
    \[
    \text{三元环数量} = \binom{n}{3} - \sum_{i=1}^n \binom{d_i}{2}
    \]
    其中 $d_i$ 是顶点 $i$ 的出度。
    \item \textbf{最大三元环数}:当各点出度尽可能接近(即 $|d_i - d_j| \le 1$)时,三元环数量最多。
    \begin{itemize}
        \item $n$ 为奇数时,最大数量为 $\frac{n(n^2-1)}{24}$(正则竞赛图)。
        \item $n$ 为偶数时,最大数量为 $\frac{n(n^2-4)}{24}$。
    \end{itemize}
    \item \textbf{线性性}:包含最多 3-环的竞赛图是正则的(或接近正则),包含最少 3-环(0 个)的竞赛图是传递的。
\end{enumerate}

\textbf{算法与构造}
\begin{enumerate}
    \item \textbf{寻找哈密顿路径} ($O(n \log n)$ 或 $O(n)$):
    \begin{itemize}
        \item \textbf{归纳法/插入法}:假设已知 $v_1 \to v_2 \to \dots \to v_k$ 是前 $k$ 个点的路径。考虑新点 $u$,若 $u \to v_1$,则插在头部;若 $v_k \to u$,插在尾部;否则存在 $i$ 使得 $v_i \to u \to v_{i+1}$,插在中间。使用二分查找可达 $O(n \log n)$。
    \end{itemize}
    \item \textbf{寻找哈密顿回路} ($O(n^2)$ 或 $O(n)$):
    \begin{itemize}
        \item 若已知强连通,先找一条哈密顿路径 $v_1 \to \dots \to v_n$。
        \item 若 $v_n \to v_1$,则构成回路。
        \item 否则,必存在 $i$ 使得 $v_i \to v_1$ 且 $v_n \to v_{i+1}$(这构成了一个 $3 \le len < n$ 的环加上路径剩余部分),可以通过调整构造出回路。
    \end{itemize}
    \item \textbf{寻找一个三元环} ($O(n)$):
    \begin{itemize}
        \item 仅判断是否存在:检查出度序列是否为 $0, 1, \dots, n-1$(传递图无环)。
        \item 构造:若存在,遍历所有点 $u$,找到出度最小的邻居 $v$( 即 $u \to v$ 且 $out(v)$ 最小)。若 $out(v) < out(u)$,则必存在 $w$ 使得 $v \to w \to u$。实际上,只需遍历边 $(u, v)$,若 $out(v) < out(u)$ 则极大概率在环中。
    \end{itemize}
\end{enumerate}

\textbf{特殊结论}
\begin{itemize}
    \item \textbf{国王 (King)}:
    \begin{itemize}
        \item \textbf{定义}:若顶点 $K$ 满足对于任意其他顶点 $v$,要么 $K \to v$,要么存在 $w$ 使得 $K \to w \to v$(即 $K$ 到任意点的距离 $\le 2$),则称 $K$ 为国王。
        \item \textbf{Landau (1953)}:任意竞赛图都至少存在一个国王。特别地,**出度最大的点一定是国王**(可能有多个出度最大的点,它们都是国王;但也可能存在出度不是最大的国王)。
        \item \textbf{数量性质}:
        \begin{itemize}
            \item 竞赛图没有“独裁者”(出度为 $n-1$ 的点)$\iff$ 原图强连通 $\iff$ 竞赛图至少有 3 个国王。
            \item 实际上,强连通竞赛图中,国王的数量 $k$ 满足 $3 \le k \le n$。例如 $n$ 为奇数的正则竞赛图,所有点都是国王。
        \end{itemize}
    \end{itemize}
    \item \textbf{1-0 矩阵性质}:竞赛图的邻接矩阵 $A$ 满足 $A + A^T = J - I$($J$ 为全1矩阵,$I$ 为单位阵)。
    \item \textbf{翻转性质}:翻转竞赛图中任意一个三元环($u \to v \to w \to u$)的方向(变为 $u \leftarrow v \leftarrow w \leftarrow u$),所有点的\textbf{出度不变}。
    \begin{itemize}
        \item \emph{解释}:在环中每个点都贡献了 1 个出度。翻转后,每个点依然贡献 1 个出度(指向原来的上家)。此性质常用于证明兰道定理的可行性:任意合法的出度序列都可以通过不断翻转三元环从传递图得到。
    \end{itemize}
\end{itemize}
