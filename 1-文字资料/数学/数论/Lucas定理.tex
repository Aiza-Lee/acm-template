\textbf{Lucas定理}

对于质数 $p$,非负整数 $n, m$,有:
\[ \binom{n}{m} \equiv \binom{\lfloor n/p \rfloor}{\lfloor m/p \rfloor} \cdot \binom{n \bmod p}{m \bmod p} \pmod p \]
即若 $n, m$ 的 $p$ 进制展开为 $n = (n_k \dots n_1 n_0)_p, m = (m_k \dots m_1 m_0)_p$,则:
\[ \binom{n}{m} \equiv \prod_{i=0}^k \binom{n_i}{m_i} \pmod p \]

\textbf{扩展 Lucas 定理}
计算 $\binom{n}{m} \pmod M$,其中 $M$ 不一定是质数。

将 $M$ 分解质因数 $M = \prod p_i^{c_i}$。
对于每个 $p_i^{c_i}$,计算 $\binom{n}{m} \pmod{p_i^{c_i}}$。
最后使用中国剩余定理合并结果。

计算 $\binom{n}{m} \pmod{p^k}$:
\[ \binom{n}{m} = \frac{n!}{m!(n-m)!} \]
不能直接求逆元,因为分母可能包含 $p$ 的因子。
计算 $n! \pmod{p^k}$ 时,先提取出所有 $p$ 的倍数。
$n! = p^{\sum_{j=1}^{\infty} \lfloor n/p^j \rfloor} \cdot \underbrace{\prod_{i=1, p \nmid i}^n i}_{\text{部分 A}} $
部分 A 也具有循环性质,循环节长度为 $p^k$,可以递归或循环计算。