\textbf{概述}

求解积性函数前缀和

\[F(n) := \sum_{i=1}^n f(i).\]

要求存在一个函数 \(g\) 满足: \(g\) 是积性函数。 \(g\) 易求前缀和

对于质数 \(p\) , \(g(p) = f(p)\) 。


\textbf{Powerful Number}

定义:对于正整数 \(n\) ,记 \(n\) 的质因数分解为 \(n=\prod_{i=1}^{m}p_i^{e_i}\) 。 \(\) \(n\) 是PN当且仅当 \(\forall 1\le i \le m, e_i > 1\) 。( \(1\) 满足PN定义)

性质:

所有的PN都可以表示为 \(a^2b^3\) 的形式。 \(n\) 以内的PN至多有 \(O(\sqrt n)\) 个。

性质2的证明:枚举 \(a\) ,再考虑满足条件的 \(b\) 的个数,则PN的个数约为:

\[\int_1^{\sqrt n} \left(\frac{n}{x^2}\right)^{\frac{1}{3}} \text d x = O(\sqrt n).\]


\textbf{PN 筛}

\[G(n) := \sum_{i=1}^{n}g(i).\]

构造函数 \(h=f/g\) ,这里的“ \(/\) ”表示狄利克雷卷积。则 \(h(1)=1\) ,且 \(f=g*h\) .

对于素数 \(p\) ,有 \(f(p) = g(1)h(p) + g(p)h(1) = h(p)+g(p)\) ,进而得到 \(h(p)=0\) ,又因为 \(h\) 是积性函数,则 \(h\) 仅在PN处取值可能不为 \(0\) .

根据 \(f=g*h\) ,有:

\[\begin{aligned}
F(n) &= \sum_{i=1}^n f(i) \\
        &= \sum_{i=1}^{n}\sum_{d | i} h(d)g\left(\frac{i}{d}\right) \\
        &= \sum_{d=1}^n h(d)\sum_{i=1}^{\lfloor\frac{n}{d}\rfloor}g(i) \\
        &= \sum_{\substack{d \in \mathbb{PN} \\ d \le n}} h(d)G(\lfloor\frac{n}{d}\rfloor).
\end{aligned}\]

现在只需要在 \(O(\sqrt n)\) 的时间复杂度找到所有PN,并计算所有 \(h\) 的有效值。其中 \(h\) 的有效值可以通过计算 \(h(p^e)\) ,然后通过 \(h\) 积性函数的性质算出所有 \(h\) 的值。

常见的计算 \(h(p^e)\) 的方法有两种:

直接算出关于 \(p,e\) 的表达式。

根据 \(f(p^e) = \sum_{i=0}^{e} g(p^i)h(p^{e-i})\) ,移项得到 \(h(p^e)=f(p^e)-\sum_{i=1}^{e}g(p^i)h(p^{e-i})\) .