\textbf{[容斥原理]}

\textbf{定理}:
设 $S$ 为有限集合,$A_1, A_2, \dots, A_n \subseteq S$,则
\[
\left| \bigcup_{i=1}^n A_i \right| = \sum_{\emptyset \neq T \subseteq \{1, \dots, n\}} (-1)^{|T|-1} \left| \bigcap_{i \in T} A_i \right|
\]
补集形式(常用于求解满足 \textbf{0个性质} 的元素个数):
\[
\left| \bigcap_{i=1}^n \overline{A_i} \right| = |S| - \left| \bigcup_{i=1}^n A_i \right| = \sum_{T \subseteq \{1, \dots, n\}} (-1)^{|T|} \left| \bigcap_{i \in T} A_i \right|
\]

\textbf{[Min-Max 容斥]}

对于集合 $S$,设 $\max(S)$ 为集合中元素的最大值,$\min(S)$ 为最小值,则:
\[
\max(S) = \sum_{\emptyset \neq T \subseteq S} (-1)^{|T|-1} \min(T)
\]
\[
\min(S) = \sum_{\emptyset \neq T \subseteq S} (-1)^{|T|-1} \max(T)
\]
推广到第 $k$ 大值 ($\max_k(S)$):
\[
\max\nolimits_k(S) = \sum_{\emptyset \neq T \subseteq S} (-1)^{|T|-k} \binom{|T|-1}{k-1} \min(T)
\]
\textbf{期望形式}:若元素值为随机变量,公式依然成立(即 $E[\max(S)] = \sum E[\min(T)] \dots$)。
常用场景:求一堆随机变量中 \textbf{所有变量都出现} 的期望时间(即最后一个变量出现的期望时间),转化为求 \textbf{至少一个变量出现} 的期望时间。

\textbf{[常用结论]}

1.  \textbf{错排问题}:$D_n = n! \sum_{k=0}^n \frac{(-1)^k}{k!} \approx \frac{n!}{e}$。
2.  \textbf{欧拉函数}:$\varphi(n) = n \prod_{p|n} (1 - \frac{1}{p})$ 本质是容斥原理。
3.  \textbf{不定方程解数}:求 $\sum x_i = S, 0 \le x_i \le b_i$,令属性 $P_i$ 为 $x_i \ge b_i + 1$,利用容斥计算不满足条件的方案数。
