\textbf{拉格朗日函数}

\textbf{问题形式}
\[\begin{aligned} 
\min_{x\in\mathbb R^n} \quad & f(x) \\
\text{s.t.} \quad & h_i(x) = 0, \quad i=1,\dots,m \\
& g_j(x) \le 0, \quad j=1,\dots,p 
\end{aligned}\]

\textbf{拉格朗日函数}
\[\mathcal L(x,\lambda,\mu) = f(x) + \sum_{i=1}^m \lambda_i h_i(x) + \sum_{j=1}^p \mu_j g_j(x)\]
其中 \(\lambda \in \mathbb R^m, \mu \in \mathbb R^p\) 为拉格朗日乘子。

\textbf{KKT条件(必要条件)}

设 \(x^*\) 为局部最优解,且满足正则性条件(如 LICQ),则存在 \(\lambda^*, \mu^*\) 使得:

\begin{itemize}
    \item \textbf{平稳性}(Stationarity):
    \[ \nabla f(x^*) + \sum_{i=1}^m \lambda_i^* \nabla h_i(x^*) + \sum_{j=1}^p \mu_j^* \nabla g_j(x^*) = 0 \]
    \item \textbf{原始可行性}(Primal Feasibility):
    \[ h_i(x^*) = 0, \quad g_j(x^*) \le 0 \]
    \item \textbf{对偶可行性}(Dual Feasibility):
    \[ \mu_j^* \ge 0 \]
    \item \textbf{互补松弛}(Complementary Slackness):
    \[ \mu_j^* g_j(x^*) = 0 \]
\end{itemize}

\textbf{对偶理论}

\textbf{对偶函数}
定义拉格朗日对偶函数 \(g: \mathbb R^m \times \mathbb R^p \to \mathbb R\):
\[ g(\lambda, \mu) = \inf_{x \in \mathcal{D}} \mathcal{L}(x, \lambda, \mu) \]
性质:\(g(\lambda, \mu)\) 是凹函数。

\textbf{对偶问题}
\[ \max_{\lambda, \mu} g(\lambda, \mu) \quad \text{s.t.} \quad \mu \ge 0 \]
设 \(d^*\) 为对偶问题最优值,\(p^*\) 为原问题最优值。

\textbf{弱对偶性}(Weak Duality)
\[ d^* \le p^* \]
恒成立。

\textbf{强对偶性}(Strong Duality)
\[ d^* = p^* \]
通常在凸优化问题满足 \textbf{Slater条件} 时成立。

\textbf{Slater条件}
若原问题为凸问题(\(f, g_j\) 凸,\(h_i\) 仿射),且存在 \(x \in \text{relint}(\mathcal{D})\) 使得:
\[ \forall i, h_i(x)=0, \quad \forall j, g_j(x) < 0 \]
则强对偶性成立。